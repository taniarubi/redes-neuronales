\documentclass[letterpaper,12pt]{article}

% Soporte para los acentos.
\usepackage[utf8]{inputenc}
\usepackage[T1]{fontenc}    
% Idioma español.
\usepackage[spanish,mexico, es-tabla]{babel}
% Soporte de símbolos adicionales (matemáticas)
\usepackage{multirow}
\usepackage{amsmath}		
\usepackage{amssymb}		
\usepackage{amsthm}
\usepackage{amsfonts}
\usepackage{latexsym}
\usepackage{enumerate}
\usepackage{ragged2e}
\usepackage{graphicx}
\usepackage{url}
\usepackage{hyperref}
% Modificamos los márgenes del documento.
\usepackage[lmargin=2cm,rmargin=2cm,top=2cm,bottom=2cm]{geometry}

\title{Facultad de Ciencias, UNAM \\ Redes Neuronales \\ 
       \textit{Proyecto final: Fake News Detection Dataset}}
\author{Rosado Cabrera Diego \\
        Rubí Rojas Tania Michelle}
\date{24 de mayo de 2020}

\begin{document}
\maketitle

\section{Introducción}

La diseminación de noticias falsas, bulos o rumores con el objetivo de 
manipular la opinión pública es un asunto que cada vez preocupa más en todo el 
mundo. 

Las \textit{fake news} nos llaman la atención por que están protagonizadas por 
instituciones o personajes públicos que han hecho o dicho algo controvertido,
o bien, relatan hechos sorprendentes. Suelen ser noticias polémicas, que provocan
la indignación de la sociedad en general. 


\section{Proyecto}
Nuestro proyecto consistirá en construir un modelo para la detección de 
\textit{fake news} usando el algoritmo de clasificación 
\textit{Passive-Aggressive} (ya que puede clasificar conjuntos masivos de datos, 
se puede implementar de forma rápida y se ha comprobado que es mejor que el 
\textit{MLP}). 

El conjunto de datos lo obtendremos de la siguiente página
\begin{center}
        \url{https://www.kaggle.com/c/fake-news/data}
\end{center}

Este es un archivo \textit{CSV} que contiene $7796$ filas con $5$ columnas. 
Donde las columnas refieren el id de la noticia, el título, el autor, el texto 
de la noticia y una etiqueta de \textit{True (0)} o \textit{False (1)}. 

Aunque existe una controversia entre cual es la \textit{verdad}, nosotros nos 
basaremos en el concepto de \textit{verdad} que fue utilizado para construir 
este conjunto de datos. 

\end{document}
