\documentclass[letterpaper,11pt]{article}

% Soporte para los acentos.
\usepackage[utf8]{inputenc}
\usepackage[T1]{fontenc}
% Idioma español.
\usepackage[spanish,mexico, es-tabla]{babel}
% Soporte de símbolos adicionales (matemáticas)
\usepackage{multirow}
\usepackage{amsmath}
\usepackage{amssymb}
\usepackage{amsthm}
\usepackage{amsfonts}
\usepackage{mathtools}
\DeclarePairedDelimiter\floor{\lfloor}{\rfloor}
\usepackage{latexsym}
\usepackage{enumerate}
\usepackage{ragged2e}
\usepackage{listings}
\usepackage{xcolor}
\usepackage{array}
% Modificamos los márgenes del documento.                                       %
\usepackage[lmargin=2cm,rmargin=2cm,top=2cm,bottom=2cm]{geometry}

\definecolor{codegreen}{rgb}{0,0.6,0}
\definecolor{codegray}{rgb}{0.5,0.5,0.5}
\definecolor{codepurple}{rgb}{0.58,0,0.82}
\definecolor{backcolour}{rgb}{0.95,0.95,0.92}

\lstdefinestyle{mystyle}{
    backgroundcolor=\color{backcolour},   
    commentstyle=\color{codegreen},
    keywordstyle=\color{magenta},
    numberstyle=\tiny\color{codegray},
    stringstyle=\color{codepurple},
    basicstyle=\ttfamily\footnotesize,
    breakatwhitespace=false,         
    breaklines=true,                 
    captionpos=b,                    
    keepspaces=true,                 
    numbers=left,                    
    numbersep=5pt,                  
    showspaces=false,                
    showstringspaces=false,
    showtabs=false,                  
    tabsize=2
}

\lstset{style=mystyle}

\title{Facultad de Ciencias, UNAM \\ Redes Neuronales \\ Tarea 2}
\author{Rubí Rojas Tania Michelle}
\date{13 de abril de 202}

\begin{document}
\maketitle

\begin{enumerate}
    % Ejercicio 1.
    \item Usando \textit{sklearn.datasets.make moons} genera un conjunto de 
    datos de la siguiente forma:
    \begin{verbatim}
        In [1]: C1, C2 = moons(random state=123, n samples=200, noise=0.1)
    \end{verbatim}

    \begin{enumerate}
        % Ejercicio 1.a
        \item Implementa la regresión logística usando el descenso gradiente 
        para clasificar $C_1$ y $C_2$. 

        % Ejercicio 1.b
        \item ¿Qué transformación de los datos ocupaste para poder hacer la
        correcta clasificación?
    \end{enumerate}

    % Ejercicio 2.
    \item Calcula la derivada de la tangente hiperbólica $\tanh$.
    
    \textsc{Solución:}
    \begin{align*}
        \frac{d}{dx} \tanh (x) 
        &= \frac{d}{dx} \left(\frac{\sinh (x)}{\cosh (x)}\right)
        && \text{definición de $\tanh (x)$} \\
        &= \frac{(\sinh' (x) \cdot \cosh (x)) - (\cosh' (x) \cdot \sinh (x))}
                {\cosh^2 (x)}
        && \text{derivative quotient rule} \\ 
        &= \frac{(\cosh (x) \cdot \cosh (x)) - (\sinh (x) \cdot \sinh (x))}
                {\cosh^2 (x)}
        && \text{$\sinh' (x) = \cosh (x)$ y $\cosh' (x) = \sinh (x)$} \\ 
        &= \frac{\cosh^2 (x) - \sinh^2 (x)}{\cosh^2 (x)}
        && \text{aritmética} \\ 
        &= \frac{1}{\cosh^2 (x)}
        && \text{$\cosh^2 (x) - \sinh^2 (x) = 1$} \\ 
        &= sech^2 (x)
        && \text{$\frac{1}{\cosh^2} = sech^2 (x)$}
    \end{align*}

    % Ejercicio 3.
    \item Usando el perceptrón multicapa visto en clase, clasifica a $C_1$ y 
    $C_2$. ¿Qué parámetros ocupaste?

    % Ejercicio 4.
    \item Con la red neuronal, vista en clase, que hace la clasificación 
    multiclase usando la función \textit{softmax}, realiza los siguiente 
    ejercicios:
    \begin{enumerate}
        % Ejercicio 4.a
        \item Encuentra la mejor arquitectura para el conjunto de Iris. 
        Justifica tu respuesta de por qué es la mejor.

        % Ejercicio 4.b
        \item Usa las funciones $\tan h$ y $\gamma$ en la capa intermedia. ¿Cuál
        funciona mejor?

        % Ejercicio 4.c
        \item Clasifica los siguientes estímulos y reporta a qué clase pertenece
        cada uno:
        \begin{itemize}
            \item $5.97 \; 4.20 \; 1.23 \; 0.25$
            \item $6.80 \; 5.00 \; 1.25 \; 1.20$
            \item $12.50 \; 9.20 \; 40.32 \; 21.55$
        \end{itemize}

        % Ejercicio 4.d
        \item ¿Te parecen correctas todas las clasificaciones? En caso de que 
        alguna no, ¿por qué? ¿cómo corregirías este error?
    \end{enumerate}

    % Ejercicio 5.
    \item ¿Qué es y cómo funciona la función de activación \textit{Radial Basis
    Function (RBF)}?

    \textsc{Solución:}
\end{enumerate}

\end{document}